//Abkuerzungen:
//\providecommand{\idealin}{dreieck nach rechts}

Date:20160412
\underline{Kommutative Algebra}
\section{Disclaimer}
Meine geehrten Leser,
dieses Skript ist die Mitschrift eines Studenten der Frankfurter Goethe Universitaet, sie ist voller Fehler und ihnen jeden Anspruch auf Korrektheit.
Der Vorlesende ist Professor Aley Kuronya doch ihm soll hieraus kein Schaden entstehen; wenn Unfug geschrieben steht seien sie versichert, es ist mein Fehler.
\section{Grundlagen}
ist das Studium kommutativer Ringe.\\
Im folgenden sei \( R ein kommutativer Ring m/ 1 \) \\
Wir studieren "Ringoperationen" also \bold{Moduln \"uber Ringen} \\
\begin{defn}[Moduln] \label{def_mod}
Ein \(R\)-Moduln ist eine abelsche Gruppe (additiv) und eine Abbildung 
\[R \underset{Ringhomomorphismus}{\to} \End_\Zz(M) \]
\end{defn}
\begin{hausaufgabe}
Beweise das \ref{def_mod} zur herk\"ommlichen Definition \"aquivalent ist. \\
Tipp: betrachte \( G \to \Aut(X) \)
\end{hausaufgabe}
\\
"Lineare Algebra \"uber Ringen" \\
\begin{bsp}
Sei \( K \)  ein K\"orper und \( X \) eine Menge / ein topologischer Raum, dann ist \( F_n (X,K ) \) ein kommutativer Ríng.
\end{bsp}
\begin{Ziel}
Wir wollen \underline{jeden Ring} \( R \) als einen Ring von Funktionen auf geometrischen R\"aumen auffassen k\"onnen. \\
\end{Ziel}
\section{Zusammenhang zwischen Algebra und Geometrie} 
\subtitle{Ansatz aus Sicht der algebraischen Geometrie}
Die Grundobjekte der algebraischen Geometrie werden im folgenden affine Variet\"aten sein. \\
\begin{defn}[Affine Variet\"aten]
Sei \( K \) ein beliebiger K\"orper (\zB \Rz ,\Cz,\( \Fz_p(X) \) ) und \( n \in \Nz  \), dann ist der Affine Raum \"uber \( K \) definiert als 
\[ \Az^n_K = \left\lbrace \left( a_1, \dots , a_n \right) \vert a_i \in K \right\rbrace \] \\
F\"ur \(n=1 \) sprechen wir von der affinen Geraden und f\"ur \(n=2 \) von der affinen Ebene.
\end{defn}
\begin{bem}
\underset{Raum ohne lineare Struktur}{\(\Az^n_k\) } \underset{Mengen gleich}{\leftrightarrow} \underset{Vektorraum}{\/k^n \) }
\\
\begin{konstr} //Konstruktion
Sei \( f \in k \left[ X_1, \dots, X_n \right] \) , dann ergibt \(f \) eine Funktion 
\begin{eqnarray}
\tilde{f} :	& \Az^n_k \to 								& k \\
			& \left( a_1 , \dots , a_n	\right) \mapsto	& f\left(a_1, \dots, a_n \right)
\end{eqnarray}
\end{konstr}
\begin{bem}
Falls \( \left\lvert k \right\rvert = \intfy \), dann gilt:
\[ f= 0 \Leftrightarrow \tilde{f}=0 \]
\end{bem}
\begin{bsp}
\(k = \Fz_p, f \left( x \right)=x^p-x \neq 0 \), aber \(f =0 \).
\end{bsp}
\begin{defn}[Affine Variet\"at / algebraische Menge]
Es sei \( S \subseteq k \left[X_1,\dots,X_n\right] \) , dann ist \( V \left( S \right) = \left\lbrace a \in \Az^n_k \vert \forall f \in S \ f\left( a \right) =0 \right\rbrace \)\\
Sei \( S = \left\lbrace f_1, \dots ,f_m \right\rbrace \Rightarrow V \left(f_1, \dots, f_m \right) = V \left( S \right) \)
\end{defn}
\begin{bsp}
\begin{itemize}
\item \( \Rz \left[ x,y \right], f\left(x,y \right)= x^2+y^2-1 \Rightarrow V \left( f \right) = \text{Einheitskreis} \)
\item Es sei \(f : \Az^n_k \to \Az^1_k \) eine polynomiale Abbildung. Dann gilt \( \Gamma_f= \left\lbrace \left(a,f\left(a \right) \right) \in \^{n+1}_k \vert a \in \Az^n_k \right\rbrace = V \left( x_{n+1} -f \left( x_1, \dots ,x_n \right) \right) \)
\item \( \emptyset = V (1) \) \\
	\( \Az^n_k = V (0) \) \\
	\( \left(a_1, \dots ,a_n \right) = V \left( x_1-a_1, \dots ,x_n-a_n \right) \) \\ // Ein Punkt
\end{itemize}
\begin{prop}
Es seien \( V,W \subseteq \Az^n_k \) algebraische Mengen. Dann sind auch \( V \cap W , V \cup W \) algebraische Mengen. \\
Pr\"azise gesagt: fuer \( S_1 , S_2 \subseteq k \left( x_1, \dots ,x_n \right) \)
\begin{itemize}
\item \( V \left( S_2 \right) \busset V \left( S_1 \right) \Leftrightarrow S_1 \subseteq S_2 \)
\item  \( V \left(S_1 \right) \cap V \left( S_2 \right) =  V \left(S_1  \cup  S_2 \right) \)
\item  \( V \left(S_1 \right) \cup V \left( S_2 \right) = V \left(S_1  S_2 \right) \), wobei \( S_1 S_2 : = \left\lbrace fg \vert f \in S_1 , g \in S_2 \right\rbrace \)
\begin{bem}
Es seien \left\lbrace S_i \vert i \in I \right\rbrace \subset k\left[x_1,\dots,x_n \right], dann gilt:
\[ \bigcap_{i \in I } V \left( S_i \right) ) = V \left( \bigcap_{i \in I } S_i \right) \]
\end{bem}
\begin{defn}[Zariski Topologie]
Die Mengen \( \tau := \left\lbrace V \left( S \right) \vert S \subseteq k \left[x_1 , \dots , x_n \right] \right\rbrace \) haben die folgenden Eigenschaften: 
\begin{itemize}
\item \( \emptyset , \Az^n_k \in \tau \) 
\item \( F,G \in \tau \Rightarrow F \cup G \)
\item \( \left\lbrace F_i \vert i \in I \right\rbrace \Rightarrow \bigcap_{i \in I} F_i \in \tau \) 
\end{itemize}
auf diese Weise haben wir eine Topologie auf \( \Az^n_k \) definiert. Sie wird als Zariski Topologie bezeichnet.
\end{defn}
\begin{bsp}
\(  \Az^3_\Rz \busseteq V \left( xy,xz \right) = V \left( x \right) \cup V \left(y,z \right) \)
//include Graphics for example here 
\begin{hausaufgabe}
Bestimmen sie alle algebraische Untermengen von \( \Az^1_k \) // \emptyset \Az^1_l \left\lbrace a \right\rbrace : a \in k 
\end{hausaufgabe}
//Hauptidealring = 1-dim geometrische Objekte
\begin{prop}
F\"ur \( S \subseteq k \left[ x_1, \dots, x_n \right] gilt
\[ V \left( S \right) = V \left( \lbrack S \rbrack \right) \]
wobei \lbrack S \rbrack das Erzeugnis von \(S \) beschreibt 
\end{prop}
\begin{defn}[Verschwindungsideal]
Sei \( T \subseteq \Az^n_k \) beliebig, dann ist das Verschwindungsideal \( I \) definiert als
\[ I \left( T \right) = \left\lbrace f \in k \left( x_1, \dots , x_ n \vert \forall a \in T f \left(a \right) = 0 \right\rbrace \] 
\end{defn}
\begin{aufgabe}
Zeige \( I ( T) \) ist ein Ideal.
\end{aufgabe}
\begin{lem}
\( T_1 , T_2 \subseteq \Az^n_k \)
\begin{itemize}
\item \( T_1 \subseteq T_2 \Rightarrow I (T_2 ) \subseteq I(T_1) \)
\item \( I \left( T_1 \cup T_2 \right) = I \left( T_1 \right) \cap I \left( T_2 \right) \)
\end{itemize}
Dies gilt auch f\"ur beliebige Indexmengen.
\end{lem}
\begin{Frage}
Kann jede algebraische Untermenge von \( \Az^n_k \) von endlich vielen Polynomen beschrieben werden.
\end{Frage}
\begin{satz}[Hilbertscher Basissatz]
Sei  \( R \) ein kommutativer Ring, dann gilt:
\[ R \text{noethersch} \Rightarrow R\left[ x \right] \text{noethersch} \]
\end{satz}
\begin{lem}
És sei \( R \) ein beliebiger Ring, dann sind \"aquivalent:
\begin{itemize}
\label{noet}
\item Jedes Ideal \( I \subseteq R \) ist endlich erzeugt
\item Jede aufsteigende Kette von Idealen stabilisiert
\end{itemize}
\end{lem}
\begin{defn}[Noethersch]
Wenn ein Ring die Eigenschaften aus \ref{noet} erf\"ullt heisst er noethersch
\end{defn}
\begin{hausaufgabe}
Rechnen sie den Beweis durch
\end{hausaufgabe}

Date:20160419
\begin{lem}
Sei \( R \) ein Ring, dann sind \"aquivalent:
\begin{itemize}
\item \( \forall I \idealin R \) ist \( I \) endlich erzeugt
\item \( R \) hat die ACC (ascending chain condition) Eigenschaft
\end{itemize}
\begin{proof}
\( (ii) \Rightarrow (i) \): Sei \( I \idealin R \) beliebig.
Man definiere die folgende Kette :
Sei \( f_i \in I , I_1 = \left( f_1 \right) \) ; 
\begin{itemize}
\item Falls \( I_1=I \), dann ist unsere Aussage bereits bewiesen
\item Falls \( I_1 \neq I \), dann existiert \( f_2 \in I \backslash I_1 \) , dann sei \( I_2:= \left( f_1 ,f_2 \right) \)
\end{itemize}
Auf diese Weise erreichen wir nach endlich vielen Schritten \( I = I_n = \left( f_1, \dots , f_n \right) \) oder es gibt eine Kette \( I_1 \subsetneq I_2 \subsetneq \dots \) was ein Widerspruch zur ACC-Eigenschaft ist.
\end{proof}
\end{lem}
\begin{satz}[Hilbertscher Basissatz]
\( R \text{noethersch} \Rightarrow R \left[ X \right] \text{noethersch} \)
\begin{proof}
Wi beweisen durch Widerspruch:
Sei \( I \idealin R\left[ X \right] \) ein Ideal, das nicht e. e. ?? ist.
Dann \( \exists \infty-\text{Folge} \) von Elementen \( f_i \in R \left[X \right] \) mit den folgenden Eigenschaften 
\begin{itemize}
\item \( f_0 \in I \backslash \left\lbrace 0 \right\rbrace \) und minimalem Grad
\item \( f_1 \in I \backslash \left( f_0 \right) \) und minimalem Grad
\item \( f_{i+1} \in I \backslash \left( f_0 , \dots , f_i \right) \) 
\end{itemize}
\underline{Merke:}
\( \forall i \geq 0 \deg f_i \leq \deg f_{i+1} \) \\
Sei \( a_i \) der Leitkoeffizient von \( f_ i \) , dann gilt \( I_i := \left( a_0, \dots , a_i \right) \idealin R \).
Aus der Konstruktion folgt:
\( I_0 \subseteq I_1 \subseteq \dots \) .
Da \( R \) noethersch ist, wird die Kette \( I_j \) stabilisieren. \\
\( \Rightarrow \exists m \in \Nz: I_m = I_{m+1} = \dots \) \\
\( \Rightarrow \forall n \in \Nz: n \geq m a_n \in \left( a_0, \dots, a_n \right) \), das heisst \( \exists r_0 , \dots r_m \in R: a_{m+1} = \sum_i=0^m r_i a_i \)  \\
Betrachte jetzt \( f:= f_{m+1} - \sum_{i=0}^m x^{\alpha_i} r_i f_i \) , wobei \( \alpha_i := \deg f_{m+1} - \deg f_i \) \\
\underline{Einerseits :} \( f \in \left( f_0, \dots , f_m \right) \), da \( f_{m+1} \in I \backslash \left( f_0, \dots, f_m \right) \).
\underline{Andererseits :} \( \deg f \le \deg f_{m+1} \) , da der Koeffizient von \( x^{\deg f_{m+1} \) in \( f =0 \).
\end{proof}
\end{satz}

\begin{hausaufgabe}
Wir haben Leitkoeffizienten von \(f_{m+1} \) zu 0 gemacht.
\end{hausaufgabe}
\begin{kor}
\begin{itemize}
\item \( k \left[ x_1 , \dots , x_n \right] \) ist noethersch
\item \( \Zz \left[ x_1 , \dtos ,x_n \right] \) ist noethersch
\end{itemize}
\end{kor}
\begin{frage}
\label{unnoeth}
Ist \( k \left[x_1, \dots , x_n, \dots \righŧ] \) noethersch?
\end{frage}
\begin{hausaufgabe}
Zeigen sie das \ref{unnoeth} nicht gilt.
\end{hausaufgabe}
\underline{Tipp:}  \( x_{n+1} \notin \left(x_1, \dots , x_n} \)
\begin{kor}
Sei  \(V =V \left( S \right) \subset \Az_k^n \Rightarrow \exists S_1 \subseteq S : \left\lvert S_1 \right\rvert \le \infty \) mit \( V \left( S_1 \right) = V \left( S \right) \)
\end{kor} 
Wir haben bereits gesehen:
\[ \left\lbrace Algebraische Mengen \subseteq \Az_k^n \right\rbrace arrow left I () / arrow right V() \left\lbrace Ideale in k \left[x_1 , \dots , x_n \right] \]
\begin{bem}
Punkte in \( \Az_k^n \) entsprechen minimalen nicht-leeren algebraischen Mengen. \\
\( \rightarrow \) Wir erwarten, dass f\"ur einen Punkt \( p \in \Az_k^n \ I (p) \idealin k \left[ x_1, \dots, x_n \right] \) maximal ist.
 
\end{satz}

