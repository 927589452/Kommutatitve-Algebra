Date:20160412
\underline{Kommutative Algebra}
\section{Disclaimer}
Meine geehrten Leser,
dieses Skript ist die Mitschrift eines Studenten der Frankfurter Goethe Universitaet, sie ist voller Fehler und ihnen jeden Anspruch auf Korrektheit.
Der Vorlesende ist Professor Aley Kuronya doch ihm soll hieraus kein Schaden entstehen; wenn Unfug geschrieben steht seien sie versichert, es ist mein Fehler.
\section{Grundlagen}
ist das Studium kommutativer Ringe.\\
Im folgenden sei \( R ein kommutativer Ring m/ 1 \) \\
Wir studieren "Ringoperationen" also \bold{Moduln \"uber Ringen} \\
\begin{defn}[Moduln] \label{def_mod}
Ein \(R\)-Moduln ist eine abelsche Gruppe (additiv) und eine Abbildung 
\[R \underset{Ringhomomorphismus}{\to} \End_\Zz(M) \]
\end{defn}
\begin{hausaufgabe}
Beweise das \ref{def_mod} zur herk\"ommlichen Definition \"aquivalent ist. \\
Tipp: betrachte \( G \to \Aut(X) \)
\end{hausaufgabe}
\\
"Lineare Algebra \"uber Ringen" \\
\begin{bsp}
Sei \( K \)  ein K\"orper und \( X \) eine Menge / ein topologischer Raum, dann ist \( F_n (X,K ) \) ein kommutativer Ríng.
\end{bsp}
\begin{Ziel}
Wir wollen \underline{jeden Ring} \( R \) als einen Ring von Funktionen auf geometrischen R\"aumen auffassen k\"onnen. \\
\end{Ziel}
\section{Zusammenhang zwischen Algebra und Geometrie} 
\subtitle{Ansatz aus Sicht der algebraischen Geometrie}
Die Grundobjekte der algebraischen Geometrie werden im folgenden affine Variet\"aten sein. \\
\begin{defn}[Affine Variet\"aten]
Sei \( K \) ein beliebiger K\"orper (\zB \Rz ,\Cz,\( \Fz_p(X) \) ) und \( n \in \Nz  \), dann ist der Affine Raum \"uber \( K \) definiert als 
\[ \Az^n_K = \left\lbrace \left( a_1, \dots , a_n \right) \vert a_i \in K \right\rbrace \] \\
F\"ur \(n=1 \) sprechen wir von der affinen Geraden und f\"ur \(n=2 \) von der affinen Ebene.
\end{defn}
\begin{bem}
\underset{Raum ohne lineare Struktur}{\(\Az^n_k\) } \underset{Mengen gleich}{\leftrightarrow} \underset{Vektorraum}{\/k^n \) }
\\
\begin{konstr} //Konstruktion
Sei \( f \in k \left[ X_1, \dots, X_n \right] \) , dann ergibt \(f \) eine Funktion 
\begin{eqnarray}
\tilde{f} :	& \Az^n_k \to 								& k \\
			& \left( a_1 , \dots , a_n	\right) \mapsto	& f\left(a_1, \dots, a_n \right)
\end{eqnarray}
\end{konstr}
\begin{bem}
Falls \( \left\lvert k \right\rvert = \intfy \), dann gilt:
\[ f= 0 \Leftrightarrow \tilde{f}=0 \]
\end{bem}
\begin{bsp}
\(k = \Fz_p, f \left( x \right)=x^p-x \neq 0 \), aber \(f =0 \).
\end{bsp}
\begin{defn}[Affine Variet\"at / algebraische Menge]
Es sei \( S \subseteq k \left[X_1,\dots,X_n\right] \) , dann ist \( V \left( S \right) = \left\lbrace a \in \Az^n_k \vert \forall f \in S \ f\left( a \right) =0 \right\rbrace \)\\
Sei \( S = \left\lbrace f_1, \dots ,f_m \right\rbrace \Rightarrow V \left(f_1, \dots, f_m \right) = V \left( S \right) \)
\end{defn}
\begin{bsp}
\begin{itemize}
\item \( \Rz \left[ x,y \right], f\left(x,y \right)= x^2+y^2-1 \Rightarrow V \left( f \right) = \text{Einheitskreis} \)
\item Es sei \(f : \Az^n_k \to \Az^1_k \) eine polynomiale Abbildung. Dann gilt \( \Gamma_f= \left\lbrace \left(a,f\left(a \right) \right) \in \^{n+1}_k \vert a \in \Az^n_k \right\rbrace = V \left( x_{n+1} -f \left( x_1, \dots ,x_n \right) \right) \)
\item \( \emptyset = V (1) \) \\
	\( \Az^n_k = V (0) \) \\
	\( \left(a_1, \dots ,a_n \right) = V \left( x_1-a_1, \dots ,x_n-a_n \right) \) \\ // Ein Punkt
\end{itemize}
\begin{prop}
Es seien \( V,W \subseteq \Az^n_k \) algebraische Mengen. Dann sind auch \( V \cap W , V \cup W \) algebraische Mengen. \\
Pr\"azise gesagt: fuer \( S_1 , S_2 \subseteq k \left( x_1, \dots ,x_n \right) \)
\begin{itemize}
\item \( V \left( S_2 \right) \busset V \left( S_1 \right) \Leftrightarrow S_1 \subseteq S_2 \)
\item  \( V \left(S_1 \right) \cap V \left( S_2 \right) =  V \left(S_1  \cup  S_2 \right) \)
\item  \( V \left(S_1 \right) \cup V \left( S_2 \right) = V \left(S_1  S_2 \right) \), wobei \( S_1 S_2 : = \left\lbrace fg \vert f \in S_1 , g \in S_2 \right\rbrace \)
\begin{bem}
Es seien \left\lbrace S_i \vert i \in I \right\rbrace \subset k\left[x_1,\dots,x_n \right]
